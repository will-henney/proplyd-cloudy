\documentclass[useAMS,usenatbib]{mn2e}
\pdfoutput=1
\usepackage[varg]{txfonts}
\usepackage{astrojournals}
\usepackage{graphicx}
\usepackage{microtype}
\usepackage{xcolor}
\usepackage{fixltx2e}
\usepackage{hyperref}
\usepackage{siunitx}
\hypersetup{colorlinks=True, linkcolor=blue!50!black, citecolor=black,
  urlcolor=blue!50!black}

\usepackage{color}

\newcommand\texttheta{\ensuremath{\theta}}
\newcommand\thC{\texttheta\textsuperscript{1}\,Ori~C}
\newcommand\elec{\ensuremath{_{\mathrm{e}}}}
\newcommand\Ion[2]{\ensuremath{\mathrm{#1\,\scriptstyle #2}}}
\newcounter{ionstage}
\newcommand{\ion}[2]{% needs to be renewcommand with aastex
  \setcounter{ionstage}{#2}%
  \Ion{#1}{\Roman{ionstage}}}
\newcommand\nii{[\ion{N}{2}]}
\newcommand\oi{[\ion{O}{1}]}
\newcommand\ha{\ensuremath{\mathrm{H\alpha}}}
\newcommand\sii{[\ion{S}{2}]}
\newcommand\oiii{[\ion{O}{3}]}
\newcommand\hii{\ion{H}{2}}

\begin{document}

\addtocounter{section}{3}
\addtocounter{table}{2}
\addtocounter{figure}{7}

\section{Photoevaporation models of HST~10}
\label{sec:phot-models-hst}
\begin{table}
  \centering
  \caption{Input parameters for example physical models of 182-413} 
  \begin{tabular}{@{\,}ll@{\,}}\hline
    Stellar spectrum:& 
    \(T_* = \SI{39000}{K}\)\\
    \citep{2006AandA...448..351S} & \(\log g = 4.1\)\\
    & \(L_* = \num{2.04e5}\,L_\odot\)\\
    Ionizing flux at proplyd:& 
    \(\Phi_{\mathrm{H}} = \SI{1.27e12}{cm^{-2} s^{-1}}\)
    \\
    Ionization front radius:& 
    \(r_0 = \SI{3.7e15}{cm}\)
    \\
    Gas-phase abundances: & 
    % He 10.98, C 8.41, N 7.85, O 8.30, \\
    % (\(12 + \log_{10} z/\mathrm{H}\)) & Ne 7.56, S 6.98, Ar 6.26, Fe 5.65
    Model~A: \citet{2004MNRAS.355..229E}\\
    & Model~B: this paper
    \\
    Dust composition: & Standard Orion \citep{1991ApJ...374..580B}\\
    \hline
  \end{tabular}
  \label{tab:model:pars}
\end{table}

\begin{figure*}
  \setkeys{Gin}{width=0.7\linewidth}
  \setlength\tabcolsep{0pt}
  \centering
  \begin{tabular}{l} 
    (a)\\
    \includegraphics[]{ratios-figure-figure0.pdf}\\
    (b)\\
    \includegraphics[]{ratios-figure-figure1.pdf}
  \end{tabular}
  \caption{Comparison between predicted and observed emission line fluxes 
    for photoevaporation models of HST~10 with 
    (a)~abundances as determined for the Orion Nebula by \protect\citet{2004MNRAS.355..229E},
    (b)~abundances determined for HST~10 in this paper.  
    The agreement in panel~(b) is much better, but there are still notable discrepancies.
  }
  \label{fig:models}
\end{figure*}


\begin{figure*}
  \includegraphics[width=0.7\linewidth]{model-structure-hst10-th30}
  \caption{%
    Model structure  as a function of radius 
    along a line at an angle of \(\theta = 30^{\circ}\) to the symmetry axis 
    for our photoevaporation Model~B.
    The radius \(r\) is measured from the center of curvature of the ionization front, 
    shown on a logarithmic scale with respect to the deepest point in the model, 
    \(r_{\mathrm{min}} = 3.5 \times 10^{15}\ \mathrm{cm}\).
    \textit{Upper panel}: Physical variables. 
    Gas velocity, \(v\), in units of \(10\ \mathrm{km~s^{-1}}\) (solid black line).
    Gas temperature, \(T\), in units of \(10^4\ \mathrm{K}\) (solid red line).
    Isothermal sound speed, \(c\), in units of  \(10\ \mathrm{km~s^{-1}}\) (dashed red line).
    Electron density, \(n_{\mathrm{e}}\), in units of \(7 \times 10^4\ \mathrm{cm^{-3}}\) (dashed black line).
    Total hydrogen density, \(n_{\mathrm{H}}\), in units of  \(7 \times 10^4\ \mathrm{cm^{-3}}\) (solid purple line).
    Electron fraction, \(n_{\mathrm{e}}/n_{\mathrm{H}}\) (solid blue line).
    Lyman limit optical depth, \(\tau_0\) (divided by 10), measured from the ionizing source, 
    which is located off the graph at \(r =  6.32 \times 10^{17}\ \mathrm{cm}\) (dashed blue line).
    Visual dust extinction optical depth, \(\tau_V\) (times 10). 
    \textit{Middle panel}: Emissivity structure for selected emission lines.  
    For each emission line, the vertical line indicates the median emission radius 
    (for which half the total line flux is emitted from gas at smaller radii and half from gas at larger radii),
    while the thick horizontal bar shows the range between the 25th and 75th quartiles of the emission,
    and the thin horizontal bar shows the range between the 10th and 90th quartiles of the emission.
    \textit{Lower panel}: Ionization structure.  
    The fraction of singly ionized ions for a selection of elements is shown by differently colored lines,
    as indicated in the key.
    Results for other angles and abundance sets are similar, 
    except with densities that scale roughly as \(\cos^{1/2}\theta\), 
    and temperatures that tend to decrease slightly with \(\theta\) and with increased metal abundances.
  }
  \label{fig:model-structure}
\end{figure*}


We have calculated dynamic photoevaporation models of HST~10, 
using the procedure outlined in \S~6 of \citet{Mesa-Delgado:2012}. 
The parameters for the models are shown in Table~\ref{tab:model:pars}. 
As compared with 177-341 (HST~1), which was modeled in \citet{Mesa-Delgado:2012},
HST~10 receives a roughly ten times smaller ionizing flux
and is roughly twice as large. 
Since \(F \propto n^2 r\) for recombination-dominated photoevaporation flows \citep{1990ApJ...354..529B, 2001RMxAC..10...57H},
this implies that the densities in HST~10 should be \(\simeq 5\) times smaller
and the ionization parameter \(\simeq 2\) times smaller than in HST~1. 

Two different abundance sets were used in the models.  
The first (Model~A) is the standard Orion gas phase abundance set, 
as determined by \cite{2004MNRAS.355..229E}.
The second (Model~B) is the set of abundances determined in this paper by empirical means, 
see Table~3. 
The resultant spectrum is shown in Figure~\ref{fig:models} for the two cases. 
It can be seen that the \citeauthor{2004MNRAS.355..229E} abundances 
produce very large discrepancies between the observed and predicted line fluxes 
(panel a). 
In particular, all Sulfur, Argon and Chlorine lines are too strong in the model 
by a factor of 3 to 10, 
whereas, \oiii{} 4363~\AA{} is too weak, indicating that the model temperature 
is too low in the highly ionized regions.
On the other hand, the \nii{} lines are well-reproduced y this model. 

The situation is much improved by using the Model~B abundances (panel b),
although serious discrepancies remain.  
The sulfur line fluxes are now in good agreement with the observations,
with the notable exception of the \sii{} 4070~\AA{} auroral line,
which is still 3 times too strong in the model. 
The \oi{} 6300~\AA{} shows a similar behaviour, being too strong by a factor of 4. 
These two lines, together with the \nii{} 5755~\AA{} auroral line, 
show the strongest contrast between the proplyd and the background nebula,
and hence are measured with a relatively small uncertainty,
making the disagreement highly significant. 
The \nii{} lines do not agree with Model~B so well as they do with Model~A, 
but the agreement is still fair given the uncertainties. 
The remaining disagreement is with the [\ion{N}{1}] 5199~\AA{} line,
but this is to be expected since the line arises through fluorescence in neutral gas
\citep{Ferland:2012},
whereas the model extends only to a hydrogen ionization fraction of 0.1 
and therefore misses part of the [\ion{N}{1}]-emitting zone.

It is not particularly surprising that the empirically determined abundances
do not reproduce the observed fluxes when used in our physical proplyd model. 
The same was seen in the case of HST~1,
where it was argued that the strong gradients in temperature and density,
both with and between ionization zones,
invalidated the empirical one-zone or two-zone approaches for analysing the spectrum.
However, in that case it was possible to find a different set of abundances for the model
that succeded in reproducing the observed spectrum to a high precision. 
Unfortunately, we have been unable to find such a satisfactory model for HST~10. 
Initial attempts to adjust the abundances have resulted in models that fit the observations 
\emph{worse} than Model~B does. 

The two most discrepant lines, \oi{}~6300~\AA{} and \sii{} 4070~\AA{} have 
a considerable overlap in their zones of emission.  
All the \oi{} emission, and roughly half of the \sii{} 4070~\AA{} arises 
from partially ionizes gas at the ionization front itself.
If the temperature of this gas is overestimated in the models, 
then the discrepancy could be explained. 
With the standard Orion dust properties that we are using in all the models
\citep{1991ApJ...374..580B},
photoelectric emission from dust grains provides up to 15\% of the heating
in this zone. 
Since there is some evidence that dust is depleted in the proplyd flows
\citep{2001ApJ...561..830G}, it may be feasible to reduce this heating,
which is an avenue that will be explored in future work. 

\section{Spatial variation}

The emissivity structure shown in the middle panel of Figure~\ref{fig:model-structure} can be compared with the emission line maps presented in \S~3.1.   The model predicts that the lower ionization lines, such as those of [\ion{O}{1}], [\ion{S}{2}], and [\ion{N}{2}], should show  emission  that is offset towards the center of the proplyd with respect to the H\(\alpha\) emission by \(1\)--\(2 \times 10^{15}\)~cm (0.5--1 spaxel), which is very similar to what is observed (Fig.~3).   In contrast, the highest ionization lines, such as those of [\ion{O}{3}] and [\ion{Ne}{3}], should be offset away from the center of the proplyd by a similar amount, which is again consistent with the observed spatial distributions (Fig.~4).    The model also makes more detailed predictions about the emission distributions, such as the fact that for both [\ion{O}{3}] and [\ion{N}{2}] the nebular collisional lines should be weighted towards slightly larger radii than the auroral line, a consequence of the temperature and density gradients within a given ionization zone.   There is some evidence for this from our emission maps, but the predicted magnitude of the effect (0.1--0.5 spaxel) is at the limit of what can be discerned with our spatial resolution.

\setcounter{section}{4}
\setcounter{subsection}{2}
\subsection{Comparison with other proplyds}
\begin{figure}
  \centering
  \includegraphics[width=\linewidth]{prop-size-2012}
  \caption{Variation with projected distance from the ionizing star of circumstellar disc sizes in the Orion proplyds.   Blue symbols show empirical determinations from \citet{Vicente:2005}, red symbols show results from fitting photoevaporation models \citep{Henney:1998, 1999AJ....118.2350H}.   The three proplyds that have been subject to gas-phase abundance studies are marked: 167-317 (LV~2), 177-341 (HST~1), a 182-413 (HST~10).   In only a few cases, such as HST~10, is the molecular disc size directly measured, for the rest it is assumed to be half the size of the ionization front.   Note that the \citeauthor{Vicente:2005} methodology significantly overestimates the true size of the ionization front (and by extension the enclosed disc) for proplyds that are much brighter than surrounding nebula, which tend to be those found close to the ionizing star.   Error bars show the mean and standard deviation measured in 4 broad spatial bins.}
  \label{fig:sizes}
\end{figure}
\begin{table*}\centering
  \caption{Comparison of physical properties between HST~10 and two other well-studied proplyds}
\label{tab:3props}
\setlength\tabcolsep{2\tabcolsep}
\begin{tabular}{llr rrr}\hline
 & Units & Note & LV~2 & HST~1 & HST~10 \\ \hline
Coordinate-based designation & & 1 & 167-317 & 177-341 & 182-413 \\
\rule{0pt}{3ex}\textit{Relation to ionizing source}\\
Projected distance, \(D'\) & \(''\) & 2 & 7.83 & 25.84 & 56.7 \\
Inclination, \(i\) & \(^\circ\) & 3 & 50 & 70 & 150 \\
True distance, \(D\) & pc & 4 & 0.022 & 0.059 & 0.242 \\
\rule{0pt}{3ex}\textit{Ionized cusp} \\
Ionization front radius, \(r_0\) & AU & 5 & 53. & 136. & 247. \\
Peak electron density, \(n_0\) & \(10^{6}\ \mathrm{cm^{-3}}\) & 6 & \(2.0\) & \(0.4\) & \(0.1\) \\
Ionization parameter &  & 7 & 0.012 & 0.008 & 0.002 \\
Cusp mass-loss rate, \(\dot{M}\) & \(10^{-7}\ M_\odot\ \mathrm{yr}^{-1}\) & 8 & \(2.6\) & \(2.5\) & \(2.1\) \\
\rule{0pt}{3ex}\textit{Molecular disc} \\
Disc effective temperature: \(T_\mathrm{d}\) & K & 9 & 95 & 58 & 29 \\
Disc mass \(M_\mathrm{d}\) & \(10^{-3}\ M_\odot\) & 10 & \(1.6\) & \(2.7\) & \(5.4\) \\
Disc radius \(R_\mathrm{d}\) & AU & 11 & 34 & 89 & 160 \\
% Sigma0(10 AU) & g/cm^2 & 7.57 & 5.06 & 5.59 \\
Evaporation time, \(t_\mathrm{evap}\) & \(10^4\ \mathrm{yr}\) & 12 & \(0.6\) & \(1.1\) & \(2.6\) \\
\hline
\end{tabular}
\par\smallskip
\begin{minipage}{0.7\linewidth}
  % \def\NoteSep{\\}
  \def\NoteSep{\quad}
  \textit{Notes}:\NoteSep 
  (1)~\citet{1994ApJ...436..194O}\NoteSep
  (2)~Angular separation from \thC{} \citep{1998AJ....115..263O}\NoteSep
  (3)~Inclination of proplyd axis to line of sight estimated from kinematic studies of the velocity--ionization correlation in emission lines from the cusp \citep{1999AJ....118.2350H, 2002ApJ...566..315H}.  Proplyds with \(i > 90^\circ\) have their head pointing away from the observer.\NoteSep
  (4)~\(D = D' / \sin i\).\NoteSep
  (5)~Estimated from fitting evaporation models to the \ha{} profiles of the cusps \cite{Henney:1998}.\NoteSep
  (6)~LV~2 from [\ion{C}{3}] density \citep{2002ApJ...566..315H}; HST~1 and HST~10 from model fitting (this paper and \citealp{Mesa-Delgado:2012}).\NoteSep
  (7)~\(F / (n_0\,c)\).\NoteSep 
  (8)~Calculated by integrating model mass fluxes over the area of the cusp.\NoteSep
  (9)~Radiative equilibrium temperature, assuming that 25\% of the bolometric flux from \thC{} reaches the surface of the disk (see also \citealp{Robberto:2002}).\NoteSep
  (10)~Estimated from observed fluxes at \(880~\mu\mathrm{m}\) \citep{Mann:2010} after subtracting the contribution from ionized free-free emission, assuming optically thin dust emission with opacity \(\kappa_\nu = 0.034~\mathrm{cm^2\ g^{-1}}\) and dust temperature equal to the effective temperatures derived above.\NoteSep
  (11)~Directly estimated from \textit{HST} images from HST~10.  For LV~2 and HST~1, we assume \(r_\mathrm{d} = 0.65 r_0\), see Figure~\ref{fig:sizes}.\NoteSep
  (12)~Nominal mass loss timescale: \(M_\mathrm{d} /\dot{M}\). 
\end{minipage}


\end{table*}

HST~10 is the third proplyd to be subject to a detailed abundance analysis, following earlier studies of LV~2 \citep{Tsamis:2011, Tsamis:2011a} and HST~1 \citep{Mesa-Delgado:2012}.  
These proplyds cover a broad range in size and in separation from the Trapezium stars (see Figure~\ref{fig:sizes}), and derived physical parameters for the three proplyds are summarised in Table~\ref{tab:model:pars}.  HST~10 shows a significantly lower ionization parameter than the two closer-in proplyds, which is reflected in its emission line spectrum that is relatively stronger in low ionization lines.  Despite these differences, the estimated mass loss rate from the ionized cusp is very similar for all three proplyds, being of order \(2 \times 10^{-7}\ M_\odot\ \mathrm{yr^{-1}}\).  These values are somewhat lower than earlier estimates (e.g., \citealp{1999AJ....118.2350H, 2002ApJ...566..315H}), which is partly because we are neglecting the contribution of mass loss though the proplyd tail.

Table~\ref{tab:3props} also shows estimates for the mass and radius of the embedded circumstellar accretion disk, which is the reservoir of mass in the proplyds.   All three proplyds show very similar sub-mm fluxes \citep{Mann:2010}, of order 20~mJy once the contribution from ionized free-free emission has been subtracted.  However, conversion of this flux to a gas mass requires knowledge of the dust opacity per unit gas mass  and dust temperature, both of which have large uncertainties \citep{Williams:2011b}.  The values in the table are calculated on the assumption that the dust temperature is the effective temperature of a disk in radiative equilibrium with the bolometric flux from the Trapezium stars,\footnote{Although the sub-mm emission is optically thin, the disks are likely to be optically thick at mid-infrared wavelengths where they emit the bulk of their radiation.   We assume that 50\% of the bolometric radiation from the Trapezium stars is absorbed in the ionized cusp of the proplyd, and that 50\% of the remainder is absorbed in the neutral photoevaporation flow, so that only 25\% reaches the surface of the disk.} and using the opacity recommended by \citep{Beckwith:1990} of \(0.1 (\nu/1000~\mathrm{GHz})~\mathrm{cm^2~g^{-1}}\), giving \(0.034~\mathrm{cm^2~g^{-1}}\) at 880~\(\mu\)m.   It must be emphasised that the derived masses are highly uncertain, since the opacity could be up to 20 times smaller if substantial grain growth up to cm-sized bodies has occured \citep{DAlessio:2001}.   Evidence for grain evolution has been found in the case of silhouette disks projected onto the Orion Nebula \citep{Miotello:2012}.   The masses given in the table are at least 6 times smaller than the ``minimum mass solar nebula'' \citep{Weidenschilling:1977a}, which is the mass within \(30\)~AU required to account for the composition of the planets in the Solar System.  

The nominal photoevaporation timescales, \(t_\mathrm{evap} = M_\mathrm{d}/\dot{M}\), are uncomfortably short compared with the estimated age of the photoionized nebula (\(\ge 10^5\)~yr; see discussion in \S~8.2.2 of \citealp{1999AJ....118.2350H}), but would come into agreement if the masses were increased by a factor of 5--10, which cannot be ruled out (see previous paragraph).   Interestingly, the evaporation timescale \(t_\mathrm{evap}\) \emph{increases} with increasing distance from the Trapezium.  Given that \(t_\mathrm{evap}\) should be roughly equal to the elapsed time since the disk photoevaporation commenced \citep{Johnstone:1998}, this is the opposite of what would be expected from a naive model of a roughly spherical \hii{} region, in which case proplyds at greater distances from the ionizing star would have entered the \hii{} region more recently.   However, evidence from both observations \citep{ODell:2009a} and numerical simulations \citep{Arthur:2011} point to the continued survival of dense clumps of molecular gas well inside the apparent boundary of the \hii{} region, in which case it is reasonable that the proplyds closest to the Trapezium might have been shielded from ultraviolet radiation until relatively recently.

The abundance results for the three proplyds are rather disparate, so that it is very hard to see any consistent trend in the results.   For instance, although we find a roughly solar oxygen abundance for HST~10 from both empirical analysis and photoevaporation model fitting, the oxygen abundance was found to be \(\simeq 3 \times\) super-solar in LV~2 \citep{Tsamis:2011} from a purely empirical analysis, with no discrepancy between collisional and recombination lines.  In contrast, \citet{Mesa-Delgado:2012} found oxygen to be \(\simeq 3 \times\) \emph{sub}-solar in HST~1, both from empirical analysis of CELs and from photoevaporation model fitting, whereas an empirical analysis of ORLs is more consistent with a solar value.   Given the wide range of characteristic ionization and density found in the three proplyds (Table~\ref{tab:3props}), it is possible that systematic errors in our abundance analysis might be contributing to this wide spread.   In order to rule out any such effects, it is vital to carry out a similar analysis on a sample of proplyds that all have \emph{similar} ionization parameters and densities.




\bibliographystyle{mn2e}
\bibliography{BibdeskLibrary}


\end{document}
