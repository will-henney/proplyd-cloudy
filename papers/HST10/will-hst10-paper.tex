\documentclass[useAMS,usenatbib]{mn2e}
\pdfoutput=1
\usepackage[varg]{txfonts}
\usepackage{astrojournals}
\usepackage{graphicx}
\usepackage{microtype}
\usepackage{xcolor}
\usepackage{fixltx2e}
\usepackage{hyperref}
\usepackage{siunitx}
\hypersetup{colorlinks=True, linkcolor=blue!50!black, citecolor=black,
  urlcolor=blue!50!black}

\usepackage{color}

\graphicspath{ {/Users/will/Dropbox/Nahiely-Will/HST10/}, } 

\newcommand\texttheta{\ensuremath{\theta}}
\newcommand\thC{\texttheta\textsuperscript{1}\,Ori~C}
\newcommand\elec{\ensuremath{_{\mathrm{e}}}}
\newcommand\Ion[2]{\ensuremath{\mathrm{#1\,\scriptstyle #2}}}
\newcounter{ionstage}
\newcommand{\ion}[2]{% needs to be renewcommand with aastex
  \setcounter{ionstage}{#2}%
  \Ion{#1}{\Roman{ionstage}}}
\newcommand\nii{\ion{N}{2}}
\newcommand\sii{\ion{S}{2}}
\newcommand\oiii{\ion{O}{3}}

\begin{document}

\section{Introduction}

PART WRITTEN BY YIANNIS

The location on the sky of HST~10 (182-413) is halfway between the inner Trapezium cluster and the Bright Bar region, at an angular separation of roughly \(1'\) to the south-south-east from \thC{} (O7V), the principal illuminating star of the nebula.  Kinematic studies of the emission from the proplyd \cite{1999AJ....118.2350H} suggest that it is situated in the foreground of the nebula, with a true separation from \thC{} of 0.2--0.3~pc.  The proplyd is larger and fainter than the proplyds found close to the Trapezium, with a less elongated and less symmetric tail.  This is in line with the general trends seen in the proplyds \cite{1998AJ....116..293B, 1998AJ....115..263O}, which can be understood in terms of a model whereby protostellar disks around the young low-mass stars in the nebula are evaporated by the ultraviolet radiation from the high-mass stars \cite{Johnstone:1998, Henney:1998}.  

\begin{figure}
  \setkeys{Gin}{width=\linewidth}
  \centering
  \includegraphics{HST10-NHO-mosaic-wide} \\
  \includegraphics{HST10-OSN-mosaic-zoom} 
  \caption{Upper panel: Large-scale view of HST~10 in the context of the Orion Nebula.  Lower panel: Zoomed view of HST~10 and its immediate environs.  }
  \label{fig:wfpc2-images}
\end{figure}
\bibliographystyle{mn2e}
\bibliography{BibdeskLibrary}


\end{document}
